\documentclass[line_length=22zw,number_of_lines=45,twocolumn]{jlreq}

\ltjdefcharrange{23}{"F0000-"FFFFF}
\ltjsetparameter{jacharrange={-2,+23}}%ギリシア文字、キリル文字を AL Char に。
\usepackage[no-math,match,deluxe,fontspec,jfm_yoko=jlreq,jfm_tate=jlreqv]{luatexja-preset}

\usepackage[unicode,colorlinks]{hyperref}
\hypersetup{linkcolor=blue,urlcolor=teal,citecolor=olive}
% \hypersetup{linkcolor=black,urlcolor=black,citecolor=black}

\usepackage{pxrubrica}
\usepackage{autobreak}
\usepackage{tikz,pgfplots,tcolorbox}
\usepackage[type=CC,modifier=by-nc,version=4.0,lang=japanese]{doclicense}

\usepackage{yhmath,amsmath,mathtools,amssymb,mathrsfs,rsfso,mleftright}
\usepackage[T1]{fontenc}
\usepackage[math]{kurier}
\usepackage[euler-digits]{eulervm}
\mleftright
\setmainfont{Exo 2}
\setsansfont{Exo 2}
\setmainjfont{FOT-MatissePro-DB}[
	AltFont={
		{
			Range={
				"4E00-"9FFF, % CJK 統合漢字
				"3400-"4DFF, % CJK 統合漢字拡張 A
				"20000-"2EBE0, % CJK 統合漢字拡張 B-F
				"2460-"24FF, % 囲み英数字
				"3200-"32FF, % 囲み CJK 文字・月
				"1F100-"1F2FF % 囲み英数字補助、漢字補助
			},
			Font=FOT-RodinNTLGPro-DB,
		},
	},
	BoldFeatures={
		Font=FOT-MatissePro-EB,
		AltFont={
			{
				Range={
					"4E00-"9FFF, % CJK 統合漢字
					"3400-"4DFF, % CJK 統合漢字拡張 A
					"20000-"2EBE0, % CJK 統合漢字拡張 B-F
					"2460-"24FF, % 囲み英数字
					"3200-"32FF, % 囲み CJK 文字・月
					"1F100-"1F2FF % 囲み英数字補助、漢字補助
				},
				Font=FOT-RodinNTLGPro-EB,
			},
		},
	},
	YokoFeatures={JFM=jlreq},   % jlreqのJFMを維持する
	TateFeatures={JFM=jlreqv},  % https://qiita.com/zr_tex8r/items/91ae1dcc9c3afce7fa8c
]
\setsansjfont{FOT-RodinNTLGPro-DB}[
	AltFont={
		{
			Range={
				"4E00-"9FFF, % CJK 統合漢字
				"3400-"4DFF, % CJK 統合漢字拡張 A
				"20000-"2EBE0, % CJK 統合漢字拡張 B-F
				"2460-"24FF, % 囲み英数字
				"3200-"32FF, % 囲み CJK 文字・月
				"1F100-"1F2FF % 囲み英数字補助、漢字補助
			},
			Font=FOT-RodinNTLGPro-DB,
		},
	},
	BoldFeatures={
		Font=FOT-MatissePro-EB,
		AltFont={
			{
				Range={
					"4E00-"9FFF, % CJK 統合漢字
					"3400-"4DFF, % CJK 統合漢字拡張 A
					"20000-"2EBE0, % CJK 統合漢字拡張 B-F
					"2460-"24FF, % 囲み英数字
					"3200-"32FF, % 囲み CJK 文字・月
					"1F100-"1F2FF % 囲み英数字補助、漢字補助
				},
				Font=FOT-RodinNTLGPro-EB,
			},
		},
	},
	YokoFeatures={JFM=jlreq},   % jlreqのJFMを維持する
	TateFeatures={JFM=jlreqv},  % https://qiita.com/zr_tex8r/items/91ae1dcc9c3afce7fa8c
]\setmonofont[
	Ligatures=TeX,
	Scale=0.89,
]{HackGen}
\setmonojfont[
	Ligatures=TeX,
	Scale=0.89,
]{HackGen}

\DeclareMathAlphabet{\mathit}{T1}{zplx}{m}{it}
\DeclareMathAlphabet{\mathtt}{T1}{fvm}{m}{n}
\DeclareMathAlphabet{\mathsf}{T1}{uop}{m}{n}
\allowdisplaybreaks[4]
\ltjenableadjust[lineend=extended,priority=true,profile=true,linestep=true]

\newcommand{\hmicoA}{{\color{violet}\jfontspec{nishiki-teki}\fontspec{nishiki-teki}\char"0262F}}
\newcommand{\hmicoB}{{\color{orange}\jfontspec{nishiki-teki}\fontspec{nishiki-teki}\char"1F3DA}}
\newcommand{\hmicoC}{{\color{magenta}\jfontspec{nishiki-teki}\fontspec{nishiki-teki}\char"1F571}}
\newcommand{\hmicoD}{{\color{olive}\jfontspec{nishiki-teki}\fontspec{nishiki-teki}\char"026F0}}
\newcommand{\hmicoE}{{\color{green}\jfontspec{nishiki-teki}\fontspec{nishiki-teki}\char"1F407}}
\newcommand{\hmicoF}{{\color{lime}\jfontspec{nishiki-teki}\fontspec{nishiki-teki}\char"1F3A9}}
\newcommand{\hmicoG}{{\color{red}\jfontspec{nishiki-teki}\fontspec{nishiki-teki}\char"1F3F0}}
\newcommand{\hmicoH}{{\color{teal}\jfontspec{nishiki-teki}\fontspec{nishiki-teki}\char"1F408}}
\newcommand{\hmicoI}{{\color{cyan}\jfontspec{nishiki-teki}\fontspec{nishiki-teki}\char"1F326}}
\newcommand{\hmicoJ}{{\color{yellow}\jfontspec{nishiki-teki}\fontspec{nishiki-teki}\char"1F318}}
\newcommand{\hmicoK}{{\color{gray}\jfontspec{nishiki-teki}\fontspec{nishiki-teki}\char"1F47B}}
\newcommand{\hmicoL}{{\color{darkgray}\jfontspec{nishiki-teki}\fontspec{nishiki-teki}\char"1F3E2}}
\newcommand{\hmicot}{{\jfontspec{nishiki-teki}\fontspec{nishiki-teki}\char"1F5E1}}
\newcommand{\hmicoo}{{\jfontspec{nishiki-teki}\fontspec{nishiki-teki}\char"1F300}}
\newcommand{\hmicoa}{{\jfontspec{nishiki-teki}\fontspec{nishiki-teki}\char"0272C}}
\newcommand{\hmicov}{{\jfontspec{nishiki-teki}\fontspec{nishiki-teki}\char"0292D}}
\newcommand{\hmicoi}{{\jfontspec{nishiki-teki}\fontspec{nishiki-teki}\char"026C8}}
\newcommand{\hmicom}{{\jfontspec{nishiki-teki}\fontspec{nishiki-teki}\char"1F570}}
\newcommand{\hmicoc}{{\jfontspec{nishiki-teki}\fontspec{nishiki-teki}\char"1F3AD}}
\newcommand{\hmicoe}{{\jfontspec{nishiki-teki}\fontspec{nishiki-teki}\char"02694}}
\newcommand{\hmicow}{{\jfontspec{nishiki-teki}\fontspec{nishiki-teki}\char"1F342}}
\newcommand{\hmicos}{{\jfontspec{nishiki-teki}\fontspec{nishiki-teki}\char"1F54B}}
\newcommand{\hmicox}{{\jfontspec{nishiki-teki}\fontspec{nishiki-teki}\char"0254D}}

\newcommand{\hmemph}[1]{\textbf{#1}}

\title{新作東方カードゲーム(タイトル未定)開発メモ}
\author{ひとみさん}

\begin{document}
\maketitle
\section{ゲーム全体に関わるメモ}
キャラクター対照表などは以下のリンクから。
\href{https://docs.google.com/spreadsheets/d/1Xvfd_cxvjWhhCuCNO1QzFNhQatAIqc_tgx4WkqzcoCo/edit?usp=sharing}
{Google ドライブ}

\begin{itemize}
	\item VISION と変える点は以下。
		\begin{itemize}
			\item 「霊力」を導入する。術者がいるとスペルカードのコストを無視できるのではなく、
				術者の霊力の合計だけ必要ノードとコストを軽減する。
			\item 他のルール変更は、総合ルールレベルでの変更に留めるつもり。
			\item リーダーやラストワードといった特殊効果については、効果を要検討。
		\end{itemize}
	\item 統率者戦みたいなことしたいよね。ということで、カードごとに「場所」を設定して、
		統率者と同じ場所のカードで 100 枚ハイランダーすれば統率者戦になるね。
		\begin{itemize}
			\item 「場所」の一覧
				\begin{itemize}
					\item \makebox[1\zw]{\hmicoA} \makebox[1em]{A} 博麗神社・八雲邸
					\item \makebox[1\zw]{\hmicoB} \makebox[1em]{B} 人間の里・命蓮寺
					\item \makebox[1\zw]{\hmicoC} \makebox[1em]{C} 魔界
					\item \makebox[1\zw]{\hmicoD} \makebox[1em]{D} 妖怪の山
					\item \makebox[1\zw]{\hmicoE} \makebox[1em]{E} 迷いの竹林・永遠亭
					\item \makebox[1\zw]{\hmicoF} \makebox[1em]{F} 魔法の森・その他・旧作
					\item \makebox[1\zw]{\hmicoG} \makebox[1em]{G} 霧の湖・紅魔館・廃洋館
					\item \makebox[1\zw]{\hmicoH} \makebox[1em]{H} 旧地獄
					\item \makebox[1\zw]{\hmicoI} \makebox[1em]{I} 仙界・天界
					\item \makebox[1\zw]{\hmicoJ} \makebox[1em]{J} 月・夢の世界
					\item \makebox[1\zw]{\hmicoK} \makebox[1em]{K} 彼岸・冥界・地獄・畜生界
					\item \makebox[1\zw]{\hmicoL} \makebox[1em]{L} 外の世界
				\end{itemize}
		\end{itemize}
		夢美さんみたいな、場所が明確な旧作勢はそこに分類して(→外の世界、魔界、地獄)、
		里香みたいな、よくわからん人たちは魔法の森に押し込む。
	\item MtG の次元カード的に、異変カードを導入しよう(ルールとしてはカジュアル選択ルール)。
		カードの内容やルールは Danmaku!!\ と同様。
	\item 様々なアイコン
		\begin{itemize}
			\item \makebox[1\zw]{\hmicot} \makebox[1em]{t} 十字
			\item \makebox[1\zw]{\hmicoo} \makebox[1em]{o} ぐるぐる
			\item \makebox[1\zw]{\hmicoa} \makebox[1em]{a} 星
			\item \makebox[1\zw]{\hmicov} \makebox[1em]{v} 複数
			\item \makebox[1\zw]{\hmicoi} \makebox[1em]{i}  瞬間
			\item \makebox[1\zw]{\hmicom} \makebox[1em]{m} 持続
			\item \makebox[1\zw]{\hmicoc} \makebox[1em]{c} 呪符
			\item \makebox[1\zw]{\hmicoe} \makebox[1em]{e} 装備
			\item \makebox[1\zw]{\hmicow} \makebox[1em]{w} 世界呪符
			\item \makebox[1\zw]{\hmicos} \makebox[1em]{s} 装備/場
			\item \makebox[1\zw]{\hmicox} \makebox[1em]{x} その他
		\end{itemize}
\end{itemize}

\section{ミーティングメモ}
\subsection{2020-06-25}
参加者: ひとみ、W・F
\begin{itemize}
	\item 冴月麟の能力、絵をどうするか(梅霖の妖精にしたほうが無難)
	\item エニグマティクドール(蓬莱人形)
	\item 人間プリズムリバーの必要性
	\item 蜃↔酒虫(案)
	\item 蓮子とメリーを Extra スターターに
	\item 玉兎を永夜抄か紺珠伝に(易者→玉兎)
	\item 神霊廟スターター: 座敷わらし→神霊
	\item 星蓮船スターターに UFO
	\item 深山の大天狗→ソクラテス
	\item ソクラテス→アリス・マーガトロイド(魔界人)
	\item タイトル案
		\begin{itemize}
			\item incarnation
			\item mirage land (有力)
		\end{itemize}
\end{itemize}

\subsubsection*{ミーティング後の動き}
\begin{itemize}
	\item Git リポジトリとしてリソースを公開。
	\item とりあえず、ミーティングで出た内容を参考にスターターのキャラクターを入れ替えた。
		(どこを入れ替えたか覚えていない。履歴をとるべきだった)
	\item 紅魔郷、天空璋+鬼形獣のキャラクターについては、6 月中に場所と種族の割当を完了
		する予定。
	\item そういえば場所のアイコンが適切かどうか聞くの忘れてたなぁ。
\end{itemize}


\section{スターターセットの内容}
どのスターターセットも、50 枚ハイランダー。
\clearpage\small
\subsection{紅魔郷スターター}
Ch: 20, SC: 20, Cm: 10
\begin{itemize}
	\item 博麗霊夢
	\item 霧雨魔理沙
	\item 冴月麟
	\item ルーミア
	\item 大妖精
	\item チルノ
	\item 紅美鈴
	\item 小悪魔
	\item パチュリー・ノーレッジ
	\item 十六夜咲夜
	\item レミリア・スカーレット
	\item フランドール・スカーレット
	\item 森近霖之助
	\item ホフゴブリン
	\item チュパカブラ
	\item 梅霖の妖精
	\item 妖精メイド
	\item ジャケットのあの子
	\item レーベルのあの子
	\item 朱鷺色の妖怪
\end{itemize}
\pagebreak

\subsection{妖々夢スターター}
Ch: 20, SC: 20, Cm: 10
\begin{itemize}
	\item レティ・ホワイトロック
	\item 橙
	\item アリス・マーガトロイド
	\item リリーホワイト
	\item リリーブラック
	\item ルナサ・プリズムリバー
	\item メルラン・プリズムリバー
	\item リリカ・プリズムリバー
	\item レイラ・プリズムリバー
	\item プリズムリバー伯爵
	\item 魂魄妖夢
	\item 魂魄妖忌
	\item 西行寺幽々子
	\item 八雲藍
	\item 八雲紫
	\item 上海人形
	\item 蓬莱人形
	\item 西行妖
	\item 死蝶霊
	\item 半幽霊
\end{itemize}
\pagebreak

\subsection{花映塚+幻想郷スターター}
Ch: 20, SC: 20, Cm: 10
\begin{itemize}
	\item 妖蓮華
	\item オレンジ
	\item くるみ
	\item 闇鏡
	\item エリー
	\item 光子
	\item 夢月
	\item 幻月
	\item メディスン・メランコリー
	\item 風見幽香
	\item 小野塚小町
	\item 四季映姫・ヤマザナドゥ
	\item サニーミルク
	\item ルナチャイルド
	\item スターサファイア
	\item 酒虫
	\item 山犬
	\item ルナサ・プリズムリバー(人間)
	\item リリカ・プリズムリバー(人間)
	\item メルラン・プリズムリバー(人間)
\end{itemize}
\pagebreak

\subsection{永夜抄スターター}
Ch: 20, SC: 20, Cm: 10
\begin{itemize}
	\item 霧雨魔理沙
	\item リグル・ナイトバグ
	\item ミスティア・ローレライ
	\item 上白沢慧音
	\item 因幡てゐ
	\item 鈴仙・優曇華院・イナバ
	\item 八意永琳
	\item 蓬莱山輝夜
	\item 藤原妹紅
	\item 綿月豊姫
	\item 綿月依姫
	\item レイセン
	\item 前鬼
	\item 後鬼
	\item 月夜見
	\item 嫦娥
	\item 瑞江浦嶋子
	\item 木花咲耶姫
	\item 岩笠
	\item 玉兎
\end{itemize}
\pagebreak

\subsection{風神録+地霊殿スターター}
Ch: 20, SC: 20, Cm: 10
\begin{itemize}
	\item 射命丸文
	\item 秋静葉
	\item 秋穣子
	\item 鍵山雛
	\item 河城にとり
	\item 犬走椛
	\item 東風谷早苗
	\item 八坂神奈子
	\item 洩矢諏訪子
	\item キスメ
	\item 黒谷ヤマメ
	\item 水橋パルスィ
	\item 星熊勇儀
	\item 古明地さとり
	\item 火焔猫燐
	\item 霊烏路空
	\item 古明地こいし
	\item 姫海棠はたて
	\item 黒猫
	\item 呪精
\end{itemize}
\pagebreak

\subsection{星蓮船+神霊廟スターター}
Ch: 20, SC: 20, Cm: 10
\begin{itemize}
	\item ナズーリン
	\item 多々良小傘
	\item 雲居一輪
	\item 雲山
	\item 村紗水蜜
	\item 寅丸星
	\item 聖白蓮
	\item 聖命蓮
	\item 封獣ぬえ
	\item 幽谷響子
	\item 宮古芳香
	\item 霍青娥
	\item 蘇我屠自古
	\item 物部布都
	\item 豊聡耳神子
	\item 二ツ岩マミゾウ
	\item 水鬼鬼神長
	\item 野鉄砲
	\item 白山修験
	\item 神霊
\end{itemize}
\pagebreak

\subsection{Extra スターター}
Ch: 20, SC: 20, Cm: 10
\begin{itemize}
	\item 伊吹萃香
	\item 永江衣玖
	\item 比那名居天子
	\item 大ナマズ
	\item 秦こころ
	\item 宇佐見菫子
	\item 依神女苑
	\item 依神紫苑
	\item 茨木華扇
	\item 宇佐見蓮子
	\item マエリベリー・ハーン
	\item 非想天則
	\item 黄帝
	\item 久米
	\item 務光
	\item 竿打
	\item 彭祖
	\item 万歳楽
	\item 龍神
	\item 蜃
\end{itemize}
\pagebreak

\subsection{輝針城+紺珠伝スターター}
Ch: 20, SC: 20, Cm: 10
\begin{itemize}
	\item わかさぎ姫
	\item 赤蛮奇
	\item 今泉影狼
	\item 九十九弁々
	\item 九十九八橋
	\item 鬼人正邪
	\item 少名針妙丸
	\item 堀川雷鼓
	\item 清蘭
	\item 鈴瑚
	\item ドレミー・スイート
	\item 稀神サグメ
	\item クラウンピース
	\item 純狐
	\item ヘカーティア・ラピスラズリ
	\item 稗田阿求
	\item 本居小鈴
	\item 管狐
	\item 運松
	\item 易者
\end{itemize}
\pagebreak

\subsection{天空璋+鬼形獣スターター}
Ch: 20, SC: 20, Cm: 10
\begin{itemize}
	\item 魂魄妖夢
	\item エタニティルラバ
	\item 坂田ネムノ
	\item 高麗野あうん
	\item 矢田寺成美
	\item 丁礼田舞
	\item 爾子田里乃
	\item 摩多羅隠岐奈
	\item 戎 瓔花
	\item 牛崎 潤美
	\item 庭渡 久侘歌
	\item 吉弔 八千慧
	\item 杖刀偶 磨弓
	\item 埴安神 袿姫
	\item 驪駒 早鬼
	\item オオカミ霊
	\item カワウソ霊
	\item オオワシ霊
	\item 饕餮
	\item 日焼けしたチルノ
\end{itemize}
\pagebreak

\subsection{封魔録+夢時空スターター}
Ch: 25, SC: 10, Cm: 15
\begin{itemize}
	\item レミリア・スカーレット
	\item フランドール・スカーレット
	\item 玄爺
	\item 神社戦車
	\item ふらわ〜戦車
	\item 里香
	\item 呪い子
	\item 明羅
	\item 幻夢盤
	\item 砲台
	\item 魔天使
	\item 飛行戦車イビルアイΣ
	\item エレン
	\item 小兎姫
	\item カナ・アナベラル
	\item 朝倉理香子
	\item 北白河ちゆり
	\item 岡崎夢美
	\item ま○ち
	\item る〜こと
	\item ミミちゃん
	\item 槌の子
	\item 煙々羅
	\item ソクラテス
	\item 先代巫女
\end{itemize}
\pagebreak

\subsection{靈異伝+怪綺談スターター}
Ch: 25, SC: 10, Cm: 15
\begin{itemize}
	\item 古明地さとり
	\item 古明地こいし
	\item アリス・マーガトロイド
	\item 博麗霊夢
	\item 神玉
	\item 幽幻魔眼
	\item 魅魔
	\item エリス
	\item キクリ
	\item サリエル
	\item 金伽羅
	\item 輪妖精
	\item サラ
	\item 神鏡
	\item ルイズ
	\item ルシア
	\item ユキ
	\item マイ
	\item 妖奈
	\item 夢子
	\item 神綺
	\item トランプキング
	\item 邪龍
	\item 沓面
	\item 蟒蛇
\end{itemize}
\pagebreak

\subsection{未割り当て}
エキスパンションなどで収録予定
\begin{itemize}
	\item ゴリアテ人形
	\item 座敷わらし
	\item ケセランパサラン
	\item 塩屋敷の旦那
	\item 京人形
	\item 露西亜人形
	\item 西藏人形
	\item オルレアン人形
	\item 仏蘭西人形
	\item 倫敦人形
	\item 和蘭人形
	\item 藁人形
	\item ティターニア
	\item 人間霊
	\item 埴輪
	\item サーヴァントフライヤー
	\item 宇治橋の怨霊
	\item 山童
	\item 天魔
	\item マミ
	\item 陰陽師
	\item いぬさくや
	\item れみにゃ
	\item ふにゃん
	\item メイベル
	\item アリオーシュ
	\item 小狐
	\item 囲碁少女
	\item お掃除メイド
	\item 神道国幻想
	\item 月光の反魂
	\item 未来世紀
	\item 蓮花の魔法使い
	\item Lotus Dream 霊夢
	\item 博麗霊夢の休日
	\item 霊夢のお茶会
	\item 最も好奇心の高い僕
	\item 最も美しいボク
	\item 最も幼い僕
	\item 最も臆病な僕
	\item 最も聡明な僕
	\item 最も大人びた僕
	\item 最も警戒心の強い僕
	\item 最も早起きな僕
	\item ビール子
	\item 仏少女
	\item トップ絵の少女
\end{itemize}

\onecolumn
\section{クレジット}
\begin{tabbing}
	\hspace{15\zw}\=\hspace{10\zw}\=\kill\\
	Total Design\>さーくる⑨\\
	Planning\>ひとみさん(さーくる⑨)\\
	Game Design\>ひとみさん(さーくる⑨)\\
	Adjusting Adviser\>SHU\\
	\>さんろ\\
	Design Adviser\>W・F(さーくる⑨)\\
	\>SHU\\
	Visual Assistant\>湯島\\
	\TeX nical Assistant\>Why Hitomisan use \TeX\ to develop this game?\\
	Illustrator\>\hmemph{募集中}\\
	\hmemph{さーくる⑨ Member}\>ひとみさん(代表)\\
	\>W・F\\
	\>Ocean@海P\\
	\>kalna\\
	\>NEO\\
\end{tabbing}

\doclicenseThis
\end{document}

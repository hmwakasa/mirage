\documentclass[line_length=22zw,number_of_lines=45,twocolumn]{jlreq}

\ltjdefcharrange{23}{"F0000-"FFFFF}
\ltjsetparameter{jacharrange={+23}}
\usepackage[no-math,match,deluxe,fontspec,jfm_yoko=jlreq,jfm_tate=jlreqv]{luatexja-preset}

\usepackage[unicode,colorlinks]{hyperref}
\hypersetup{linkcolor=blue,urlcolor=teal,citecolor=olive}
% \hypersetup{linkcolor=black,urlcolor=black,citecolor=black}

\usepackage{pxrubrica}
\usepackage{autobreak}
\usepackage{tikz,pgfplots,tcolorbox}
\usepackage[type=CC,modifier=by-nc,version=4.0,lang=japanese]{doclicense}

\usepackage{yhmath,amsmath,mathtools,amssymb,mathrsfs,rsfso,mleftright}
\usepackage[T1]{fontenc}
\usepackage[math]{kurier}
\usepackage[euler-digits]{eulervm}
\mleftright
\setmainfont{Hall Fetica Decompose}[
	BoldFont=Hall Fetica,
	ItalicFont=Hall Fetica Decompose Italic,
	BoldItalicFont=Hall Fetica Italic
]
\setsansfont{Hall Fetica Decompose}[
	BoldFont=Hall Fetica,
	ItalicFont=Hall Fetica Decompose Italic,
	BoldItalicFont=Hall Fetica Italic
]
\setmainjfont{FOT-MatissePro-M}[
	AltFont={
		{
			Range={
				"4E00-"9FFF, % CJK 統合漢字
				"3400-"4DFF, % CJK 統合漢字拡張 A
				"20000-"2EBE0, % CJK 統合漢字拡張 B-F
				"2460-"24FF, % 囲み英数字
				"3200-"32FF, % 囲み CJK 文字・月
				"1F100-"1F2FF % 囲み英数字補助、漢字補助
			},
			Font=FOT-RodinNTLGPro-M,
		},
	},
	BoldFeatures={
		Font=FOT-MatissePro-EB,
		AltFont={
			{
				Range={
					"4E00-"9FFF, % CJK 統合漢字
					"3400-"4DFF, % CJK 統合漢字拡張 A
					"20000-"2EBE0, % CJK 統合漢字拡張 B-F
					"2460-"24FF, % 囲み英数字
					"3200-"32FF, % 囲み CJK 文字・月
					"1F100-"1F2FF % 囲み英数字補助、漢字補助
				},
				Font=FOT-RodinNTLGPro-EB,
			},
		},
	},
	YokoFeatures={JFM=jlreq},   % jlreqのJFMを維持する
	TateFeatures={JFM=jlreqv},  % https://qiita.com/zr_tex8r/items/91ae1dcc9c3afce7fa8c
]
\setsansjfont{FOT-RodinNTLGPro-M}[
	AltFont={
		{
			Range={
				"4E00-"9FFF, % CJK 統合漢字
				"3400-"4DFF, % CJK 統合漢字拡張 A
				"20000-"2EBE0, % CJK 統合漢字拡張 B-F
				"2460-"24FF, % 囲み英数字
				"3200-"32FF, % 囲み CJK 文字・月
				"1F100-"1F2FF % 囲み英数字補助、漢字補助
			},
			Font=FOT-RodinNTLGPro-M,
		},
	},
	BoldFeatures={
		Font=FOT-MatissePro-EB,
		AltFont={
			{
				Range={
					"4E00-"9FFF, % CJK 統合漢字
					"3400-"4DFF, % CJK 統合漢字拡張 A
					"20000-"2EBE0, % CJK 統合漢字拡張 B-F
					"2460-"24FF, % 囲み英数字
					"3200-"32FF, % 囲み CJK 文字・月
					"1F100-"1F2FF % 囲み英数字補助、漢字補助
				},
				Font=FOT-RodinNTLGPro-EB,
			},
		},
	},
	YokoFeatures={JFM=jlreq},   % jlreqのJFMを維持する
	TateFeatures={JFM=jlreqv},  % https://qiita.com/zr_tex8r/items/91ae1dcc9c3afce7fa8c
]
\setmonofont{HackGen}[
	Ligatures=TeX,
	Scale=0.94
]
\setmonojfont{HackGen}[
	Ligatures=TeX,
	Scale=0.94
]

\DeclareMathAlphabet{\mathit}{T1}{zplx}{m}{it}
\DeclareMathAlphabet{\mathtt}{T1}{fvm}{m}{n}
\DeclareMathAlphabet{\mathsf}{T1}{uop}{m}{n}
\allowdisplaybreaks[4]
\ltjenableadjust[lineend=extended,priority=true,profile=true,linestep=true]

\colorlet{nodecolor}{violet!90!blue}
\colorlet{costcolor}{olive!30!black}
\colorlet{spiritcolor}{red!70!black}
\colorlet{grazecolor}{yellow!70!black}
\colorlet{rangebcolor}{black!90}
\colorlet{rangecolor}{black!10}
\colorlet{characolor}{teal!60}
\colorlet{spellcolor}{orange!60}
\colorlet{commandcolor}{gray!60}
\colorlet{incidentcolor}{magenta!70!black}
\colorlet{gridcolor}{cyan!60}
\definecolor{richblack}{cmyk}{.2,.2,.2,1}

\colorlet{Aicocolor}{violet}
\colorlet{Bicocolor}{orange}
\colorlet{Cicocolor}{magenta}
\colorlet{Dicocolor}{olive}
\colorlet{Eicocolor}{green}
\colorlet{Ficocolor}{lime}
\colorlet{Gicocolor}{red}
\colorlet{Hicocolor}{teal}
\colorlet{Iicocolor}{cyan}
\colorlet{Licocolor}{yellow}
\colorlet{Jicocolor}{gray}
\colorlet{Licocolor}{darkgray}

\newcommand{\Ahmico}{{\color{Aicocolor}\jfontspec{nishiki-teki}\fontspec{nishiki-teki}\char"0262F}}
\newcommand{\Bhmico}{{\color{Bicocolor}\jfontspec{nishiki-teki}\fontspec{nishiki-teki}\char"1F3DA}}
\newcommand{\Chmico}{{\color{Cicocolor}\jfontspec{nishiki-teki}\fontspec{nishiki-teki}\char"1F571}}
\newcommand{\Dhmico}{{\color{Dicocolor}\jfontspec{nishiki-teki}\fontspec{nishiki-teki}\char"026F0}}
\newcommand{\Ehmico}{{\color{Eicocolor}\jfontspec{nishiki-teki}\fontspec{nishiki-teki}\char"1F407}}
\newcommand{\Fhmico}{{\color{Ficocolor}\jfontspec{nishiki-teki}\fontspec{nishiki-teki}\char"1F3A9}}
\newcommand{\Ghmico}{{\color{Gicocolor}\jfontspec{nishiki-teki}\fontspec{nishiki-teki}\char"1F3F0}}
\newcommand{\Hhmico}{{\color{Hicocolor}\jfontspec{nishiki-teki}\fontspec{nishiki-teki}\char"1F408}}
\newcommand{\Ihmico}{{\color{Iicocolor}\jfontspec{nishiki-teki}\fontspec{nishiki-teki}\char"1F326}}
\newcommand{\Jhmico}{{\color{Licocolor}\jfontspec{nishiki-teki}\fontspec{nishiki-teki}\char"1F318}}
\newcommand{\Khmico}{{\color{Jicocolor}\jfontspec{nishiki-teki}\fontspec{nishiki-teki}\char"1F47B}}
\newcommand{\Lhmico}{{\color{Licocolor}\jfontspec{nishiki-teki}\fontspec{nishiki-teki}\char"1F3E2}}
\newcommand{\thmico}{{\jfontspec{nishiki-teki}\fontspec{nishiki-teki}\char"1F5E1}}
\newcommand{\ohmico}{{\jfontspec{nishiki-teki}\fontspec{nishiki-teki}\char"1F300}}
\newcommand{\ahmico}{{\jfontspec{nishiki-teki}\fontspec{nishiki-teki}\char"0272C}}
\newcommand{\vhmico}{{\jfontspec{nishiki-teki}\fontspec{nishiki-teki}\char"0292D}}
\newcommand{\ihmico}{{\jfontspec{nishiki-teki}\fontspec{nishiki-teki}\char"026C8}}
\newcommand{\mhmico}{{\jfontspec{nishiki-teki}\fontspec{nishiki-teki}\char"1F570}}
\newcommand{\chmico}{{\jfontspec{nishiki-teki}\fontspec{nishiki-teki}\char"1F3AD}}
\newcommand{\ehmico}{{\jfontspec{nishiki-teki}\fontspec{nishiki-teki}\char"02694}}
\newcommand{\whmico}{{\jfontspec{nishiki-teki}\fontspec{nishiki-teki}\char"1F342}}
\newcommand{\shmico}{{\jfontspec{nishiki-teki}\fontspec{nishiki-teki}\char"1F54B}}
\newcommand{\xhmico}{{\jfontspec{nishiki-teki}\fontspec{nishiki-teki}\char"0254D}}

\newcommand{\hmemph}[1]{\textbf{#1}}

\title{新作東方カードゲーム(タイトル未定)開発メモ}
\author{ひとみさん}

\begin{document}
\maketitle
\section{ゲーム全体に関わるメモ}
キャラクター対照表などは以下のリンクから。
\href{https://docs.google.com/spreadsheets/d/1Xvfd_cxvjWhhCuCNO1QzFNhQatAIqc_tgx4WkqzcoCo/edit?usp=sharing}
{Google ドライブ}

\begin{itemize}
	\item VISION と変える点は以下。
		\begin{itemize}
			\item 「霊力」を導入する。術者がいるとスペルカードのコストを無視できるのではなく、
				術者の霊力の合計だけ必要ノードとコストを軽減する。
			\item 他のルール変更は、総合ルールレベルでの変更に留めるつもり。
			\item リーダーやラストワードといった特殊効果については、効果を要検討。
		\end{itemize}
	\item 統率者戦みたいなことしたいよね。ということで、カードごとに「場所」を設定して、
		統率者と同じ場所のカードで 100 枚ハイランダーすれば統率者戦になるね。
		\begin{itemize}
			\item 「場所」の一覧
				\begin{itemize}
					\item \makebox[1\zw]{\Ahmico} \makebox[1em]{A} 博麗神社・八雲邸
					\item \makebox[1\zw]{\Bhmico} \makebox[1em]{B} 人間の里・命蓮寺
					\item \makebox[1\zw]{\Chmico} \makebox[1em]{C} 魔界
					\item \makebox[1\zw]{\Dhmico} \makebox[1em]{D} 妖怪の山
					\item \makebox[1\zw]{\Ehmico} \makebox[1em]{E} 迷いの竹林・永遠亭
					\item \makebox[1\zw]{\Fhmico} \makebox[1em]{F} 魔法の森・その他・旧作
					\item \makebox[1\zw]{\Ghmico} \makebox[1em]{G} 霧の湖・紅魔館・廃洋館
					\item \makebox[1\zw]{\Hhmico} \makebox[1em]{H} 旧地獄
					\item \makebox[1\zw]{\Ihmico} \makebox[1em]{I} 仙界・天界
					\item \makebox[1\zw]{\Jhmico} \makebox[1em]{J} 月・夢の世界
					\item \makebox[1\zw]{\Khmico} \makebox[1em]{K} 彼岸・冥界・地獄・畜生界
					\item \makebox[1\zw]{\Lhmico} \makebox[1em]{L} 外の世界
				\end{itemize}
		\end{itemize}
		夢美さんみたいな、場所が明確な旧作勢はそこに分類して(→外の世界、魔界、地獄)、
		里香みたいな、よくわからん人たちは魔法の森に押し込む。
	\item MtG の次元カード的に、異変カードを導入しよう(ルールとしてはカジュアル選択ルール)。
		カードの内容やルールは Danmaku!!\ と同様。
	\item 様々なアイコン
		\begin{itemize}
			\item \makebox[1\zw]{\thmico} \makebox[1em]{t} 十字
			\item \makebox[1\zw]{\ohmico} \makebox[1em]{o} ぐるぐる
			\item \makebox[1\zw]{\ahmico} \makebox[1em]{a} 星
			\item \makebox[1\zw]{\vhmico} \makebox[1em]{v} 複数
			\item \makebox[1\zw]{\ihmico} \makebox[1em]{i}  瞬間
			\item \makebox[1\zw]{\mhmico} \makebox[1em]{m} 持続
			\item \makebox[1\zw]{\chmico} \makebox[1em]{c} 呪符
			\item \makebox[1\zw]{\ehmico} \makebox[1em]{e} 装備
			\item \makebox[1\zw]{\whmico} \makebox[1em]{w} 世界呪符
			\item \makebox[1\zw]{\shmico} \makebox[1em]{s} 装備/場
			\item \makebox[1\zw]{\xhmico} \makebox[1em]{x} その他
		\end{itemize}
\end{itemize}

\section{印刷・出版に関するメモ}
\subsection{印刷}
\begin{itemize}
	\item 萬印堂 \url{https://www.mnd.co.jp} に頼む?
\end{itemize}

\section{カード画像の自動生成に関するメモ}
\subsection{自動生成の方法}
\begin{enumerate}
	\item カードの情報を記したファイルを作成する
		(\jruby[g]{制御綴}{コントロールシークエンス}に情報を格納する)
	\item \texttt{./make.sh fillname} を実行する
	\item PDF が自動で生成される
\end{enumerate}

\subsection{\jruby[g]{制御綴}{コントロールシークエンス}一覧}
\begin{tabbing}
	\hspace{.4\linewidth}\=\kill
	\verb+\hmmimg+\>イラスト\\
	\verb+\hmmfullimg+\>フルアートのイラスト\\
	\verb+\hmmname+\>カード名\\
	\verb+\hmmnamer+\>カード名のルビ\\
	\verb+\hmmtype+\>カードの種類 [Ch, SC, Cm, In]\\
	\verb+\hmmback+\>カードの背景 (空白でデフォルト)\\
	\verb+\hmmnode+\>必要ノード\\
	\verb+\hmmcost+\>コスト\\
	\verb+\hmmspellist+\>術者\\
	\verb+\hmmgraze+\>グレイズ\\
	\verb+\hmmspirit+\>霊力\\
	\verb+\hmmcollect+\>収集条件\\
	\verb+\hmmrangea+\>発動範囲 [toavx]\\
	\verb+\hmmranget+\>発動期間 [imcewsx]\\
	\verb+\hmmrace+\>種族\\
	\verb+\hmmplace+\>場所(複数には未対応)[A--L]\\
	\verb+\hmmtext+\>テキスト\\
	\verb+\hmmp+\>攻撃力\\
	\verb+\hmmt+\>耐久力\\
	\verb+\hmmflavor+\>フレーバーテキスト\\
	\verb+\hmmcname+\>変身状態(名称)\\
	\verb+\hmmcrace+\>変身状態(種族)\\
	\verb+\hmmcspirit+\>変身状態(霊力)\\
	\verb+\hmmcgraze+\>変身状態(グレイズ)\\
	\verb+\hmmcp+\>変身状態(攻撃力)\\
	\verb+\hmmct+\>変身状態(耐久力)\\
	\verb+\hmmctext+\>変身状態(テキスト)\\
	\verb+\hmmnumtype+\>カード番号の接頭辞\\
	\verb+\hmmnum+\>カード番号\\
	\verb+\hmmexpantion+\>エキスパンションマーク\\
	\verb+\hmmill+\>イラストレーター\\
\end{tabbing}

\section{ミーティングメモ・作業記録}
\subsection{2020-06-25}
参加者: ひとみ、W・F
\begin{itemize}
	\item 冴月麟の能力、絵をどうするか(梅霖の妖精にしたほうが無難)
	\item エニグマティクドール(蓬莱人形)
	\item 人間プリズムリバーの必要性
	\item 蜃↔酒虫(案)
	\item 蓮子とメリーを Extra スターターに
	\item 玉兎を永夜抄か紺珠伝に(易者→玉兎)
	\item 神霊廟スターター: 座敷わらし→神霊
	\item 星蓮船スターターに UFO
	\item 深山の大天狗→ソクラテス
	\item ソクラテス→アリス・マーガトロイド(魔界人)
	\item タイトル案
		\begin{itemize}
			\item incarnation
			\item mirage land (有力)
		\end{itemize}
\end{itemize}

\subsubsection*{ミーティング後の動き}
\begin{itemize}
	\item Git リポジトリとしてリソースを公開。
	\item とりあえず、ミーティングで出た内容を参考にスターターのキャラクターを入れ替えた。
		(どこを入れ替えたか覚えていない。履歴をとるべきだった)
	\item 紅魔郷、天空璋+鬼形獣のキャラクターについては、6 月中に場所と種族の割当を完了
		する予定。
	\item そういえば場所のアイコンが適切かどうか聞くの忘れてたなぁ。
\end{itemize}

\subsection{2020-07-10}
\begin{itemize}
	\item スペルカードの割当をした。
	\item コマンドカードについては、効果考えてから名前充てるのでいいや。
	\item 靈異伝スターター霊夢と封魔録スターター煙々羅を入れ替えた。
	\item 特殊カードの背景を指定できるようにした。
\end{itemize}

\subsection{2020-08-19}
\begin{itemize}
	\item 紅魔スターターのキャラクターを一通り作ってみた
	\item PR. 1 レミリア・スカーレットのドラフト版を作成した
	\item テンプレートの不具合を修正
\end{itemize}

\section{スターターセットの内容}
どのスターターセットも、50 枚ハイランダー。
\clearpage\small
\subsection{紅魔郷スターター}
Ch: 20, SC: 20, Cm: 10
\begin{itemize}
	\item キャラクターカード
		\begin{enumerate}
			\item 博麗霊夢
			\item 霧雨魔理沙
			\item 冴月麟
			\item ルーミア
			\item 大妖精
			\item チルノ
			\item 紅美鈴
			\item 小悪魔
			\item パチュリー・ノーレッジ
			\item 十六夜咲夜
			\item レミリア・スカーレット
			\item フランドール・スカーレット
			\item 森近霖之助
			\item ホフゴブリン
			\item チュパカブラ
			\item 梅霖の妖精
			\item 妖精メイド
			\item ジャケットのあの子
			\item レーベルのあの子
			\item 朱鷺色の妖怪
		\end{enumerate}
	\item スペルカード
		\begin{enumerate}
			\item 霊符「夢想封印」
			\item 夢符「封魔陣」
			\item 魔符「スターダストレヴァリエ」
			\item 恋符「マスタースパーク」
			\item 夜符「ナイトバード」
			\item 闇符「ディマーケイション」
			\item 氷符「アイシクルフォール」
			\item 凍符「パーフェクトフリーズ」
			\item 華符「芳華絢爛」
			\item 彩符「極彩颱風」
			\item 木符「シルフィホルン」
			\item 火符「アグニシャイン」
			\item 水符「プリンセスウンディネ」
			\item 奇術「ミスディレクション」
			\item 幻象「ルナクロック」
			\item 幻世「ザ・ワールド」
			\item 天罰「スターオブダビデ」
			\item 「紅色の幻想郷」
			\item 秘弾「そして誰もいなくなるか?」
			\item QED「495 年の波紋」
		\end{enumerate}
	\item コマンドカード
		\begin{enumerate}
			\item 紅魔館
			\item 霧の湖
			\item 博麗神社
			\item 
			\item 
			\item 
			\item 
			\item 
			\item 
			\item 
		\end{enumerate}
\end{itemize}
\pagebreak

\subsection{妖々夢スターター}
Ch: 20, SC: 20, Cm: 10
\begin{itemize}
	\item キャラクターカード
		\begin{enumerate}
			\item レティ・ホワイトロック
			\item 橙
			\item アリス・マーガトロイド
			\item リリーホワイト
			\item リリーブラック
			\item ルナサ・プリズムリバー
			\item メルラン・プリズムリバー
			\item リリカ・プリズムリバー
			\item レイラ・プリズムリバー
			\item プリズムリバー伯爵
			\item 魂魄妖夢
			\item 魂魄妖忌
			\item 西行寺幽々子
			\item 八雲藍
			\item 八雲紫
			\item 上海人形
			\item 蓬莱人形
			\item 西行妖
			\item 死蝶霊
			\item 半幽霊
		\end{enumerate}
	\item スペルカード
		\begin{enumerate}
			\item 寒符「リンガリングコールド」
			\item 冬符「フラワーウィザラウェイ」
			\item 仙符「鳳凰卵」
			\item 鬼神「飛翔毘沙門天」
			\item 春符「サプライズスプリング」
			\item 咒詛「魔彩光の上海人形」
			\item 咒詛「首吊り蓬莱人形」
			\item 冥鍵「ファツィオーリ冥奏」
			\item 神弦「ストラディヴァリウス」
			\item 管霊「ヒノファンタズム」
			\item 合葬「プリズムコンチェルト」
			\item 幽鬼剣「妖童餓鬼の断食」
			\item 人界剣「悟入幻想」
			\item 華霊「ゴーストバタフライ」
			\item 桜符「完全なる墨染の桜 -開花-」
			\item 式神「橙」
			\item 式神「仙狐思念」
			\item 式神「八雲藍」
			\item 結界「夢と現の呪」
			\item 紫奥義「弾幕結界」
		\end{enumerate}
	\item コマンドカード
		\begin{enumerate}
			\item 
			\item 
			\item 
			\item 
			\item 
			\item 
			\item 
			\item 
			\item 
			\item 
		\end{enumerate}
\end{itemize}
\pagebreak

\subsection{花映塚+幻想郷スターター}
Ch: 20, SC: 20, Cm: 10
\begin{itemize}
	\item キャラクターカード
		\begin{enumerate}
			\item 妖蓮華
			\item オレンジ
			\item くるみ
			\item 闇鏡
			\item エリー
			\item 光子
			\item 夢月
			\item 幻月
			\item メディスン・メランコリー
			\item 風見幽香
			\item 小野塚小町
			\item 四季映姫・ヤマザナドゥ
			\item サニーミルク
			\item ルナチャイルド
			\item スターサファイア
			\item 酒虫
			\item 山犬
			\item ルナサ・プリズムリバー(人間)
			\item リリカ・プリズムリバー(人間)
			\item メルラン・プリズムリバー(人間)
		\end{enumerate}
	\item スペルカード
		\begin{enumerate}
			\item 騒符「リリカ・ソロライブ」
			\item 騒符「メルラン・ハッピーライブ」
			\item 騒符「ルナサ・ソロライブ」
			\item 毒符「神経の毒」
			\item 毒符「憂鬱の毒」
			\item 毒符「ポイズンブレス」
			\item 花符「幻想郷の開花」
			\item 幻想「花鳥風月、嘯風弄月」
			\item 投銭「宵越しの銭」
			\item 死神「ヒガンルトゥール」
			\item 霊符「古き地縛霊の目覚め」
			\item 罪符「彷徨える大罪」
			\item 審判「ラストジャッジメント」
			\item 審判「十王裁判」
			\item 陽光「サンシャインブラスト」
			\item 光精「ダイアモンドリング」
			\item 月符「ルナティックレイン」
			\item 障光「ムーンライトウォール」
			\item 星粒「スターピースシャワー」
			\item 流星「コメットストリーム」
		\end{enumerate}
	\item コマンドカード
		\begin{enumerate}
			\item 
			\item 
			\item 
			\item 
			\item 
			\item 
			\item 
			\item 
			\item 
			\item 
		\end{enumerate}
\end{itemize}
\pagebreak

\subsection{永夜抄スターター}
Ch: 20, SC: 20, Cm: 10
\begin{itemize}
	\item キャラクターカード
		\begin{enumerate}
			\item 霧雨魔理沙
			\item リグル・ナイトバグ
			\item ミスティア・ローレライ
			\item 上白沢慧音(変身: 白沢)
			\item 因幡てゐ
			\item 鈴仙・優曇華院・イナバ
			\item 八意永琳
			\item 蓬莱山輝夜
			\item 藤原妹紅
			\item 綿月豊姫
			\item 綿月依姫
			\item レイセン
			\item 前鬼
			\item 後鬼
			\item 月夜見
			\item 嫦娥
			\item 瑞江浦嶋子
			\item 木花咲耶姫
			\item 岩笠
			\item 玉兎
		\end{enumerate}
	\item スペルカード
		\begin{enumerate}
			\item 魔空「アステロイドベルト」
			\item 黒魔「イベントホライズン」
			\item 蛍符「地上の流星」
			\item 蠢符「リトルバグ」
			\item 声符「梟の夜鳴声」
			\item 夜盲「夜雀の歌」
			\item 産霊「ファーストピラミッド」
			\item 野符「武烈クライシス」
			\item 波符「赤眼催眠(マインドシェイカー)」
			\item 狂符「幻視調律(ビジョナリチューニング)」
			\item 覚神「神代の記憶」
			\item 蘇活「生命遊戯 -ライフゲーム-」
			\item 操神「オモイカネディバイス」
			\item 難題「仏の御石の鉢 -砕けぬ意思-」
			\item 難題「火鼠の皮衣 -焦れぬ心-」
			\item 難題「蓬莱の弾の枝 -虹色の弾幕-」
			\item 転世「一条戻り橋」
			\item 時効「月のいはかさの呪い」
			\item 滅罪「正直者の死」
			\item 不滅「フェニックスの尾」
		\end{enumerate}
	\item コマンドカード
		\begin{enumerate}
			\item 
			\item 
			\item 
			\item 
			\item 
			\item 
			\item 
			\item 
			\item 
			\item 
		\end{enumerate}
\end{itemize}
\pagebreak

\subsection{風神録+地霊殿スターター}
Ch: 20, SC: 20, Cm: 10
\begin{itemize}
	\item キャラクターカード
		\begin{enumerate}
			\item 射命丸文
			\item 秋静葉
			\item 秋穣子
			\item 鍵山雛
			\item 河城にとり
			\item 犬走椛
			\item 東風谷早苗
			\item 八坂神奈子
			\item 洩矢諏訪子
			\item キスメ
			\item 黒谷ヤマメ
			\item 水橋パルスィ
			\item 星熊勇儀
			\item 古明地さとり
			\item 火焔猫燐
			\item 霊烏路空
			\item 古明地こいし
			\item 姫海棠はたて
			\item 黒猫
			\item 呪精
		\end{enumerate}
	\item スペルカード
		\begin{enumerate}
			\item 葉符「狂いの落葉」
			\item 秋符「オータムスカイ」
			\item 厄符「バッドフォーチュン」
			\item 光学「オプティカルカモフラージュ」
			\item 岐符「天の八衢」
			\item 秘術「グレイソーマタージ」
			\item 秘法「九字刺し」
			\item 神祭「エクスパンデッド・オンバシラ」
			\item 「マウンテン・オブ・フェイス」
			\item 開宴「二拝二拍一拝」
			\item 祟符「ミシャグジさま」
			\item 怪奇「釣瓶落としの怪」
			\item 罠符「キャプチャーウェブ」
			\item 妬符「グリーンアイドモンスター」
			\item 鬼符「怪力乱神」
			\item 想起「テリブルスーヴニール」
			\item 猫符「キャッツウォーク」
			\item 核熱「ニュークリアフュージョン」
			\item 表象「夢枕にご先祖総立ち」
			\item 取材「姫海棠はたての練習取材」
		\end{enumerate}
	\item コマンドカード
		\begin{enumerate}
			\item 
			\item 
			\item 
			\item 
			\item 
			\item 
			\item 
			\item 
			\item 
			\item 
		\end{enumerate}
\end{itemize}
\pagebreak

\subsection{星蓮船+神霊廟スターター}
Ch: 20, SC: 20, Cm: 10
\begin{itemize}
	\item キャラクターカード
		\begin{enumerate}
			\item ナズーリン
			\item 多々良小傘
			\item 雲居一輪
			\item 雲山
			\item 村紗水蜜
			\item 寅丸星
			\item 聖白蓮
			\item 聖命蓮
			\item 封獣ぬえ
			\item 幽谷響子
			\item 宮古芳香
			\item 霍青娥
			\item 蘇我屠自古
			\item 物部布都
			\item 豊聡耳神子
			\item 二ツ岩マミゾウ
			\item 水鬼鬼神長
			\item 野鉄砲
			\item 白山修験
			\item 神霊
		\end{enumerate}
	\item スペルカード
		\begin{enumerate}
			\item 棒符「ビジーロッド」
			\item 大輪「からかさ後光」
			\item 驚雨「ゲリラ台風」
			\item 鉄拳「問答無用の妖怪拳」
			\item 転覆「道連れアンカー」
			\item 宝塔「レイディアントトレジャー」
			\item 魔法「紫雲のオーメン」
			\item 超人「聖白蓮」
			\item 妖雲「平安のダーククラウド」
			\item 正体不明「恐怖の虹色 UFO 襲来」
			\item 響符「マウンテンエコー」
			\item 回復「ヒールバイデザイア」
			\item 邪符「ヤンシャオグイ」
			\item 雷矢「ガゴウジサイクロン」
			\item 天符「雨の磐舟」
			\item 聖童女「大物忌正餐」
			\item 名誉「十二階の色彩」
			\item 仙符「日出ずる処の天子」
			\item 壱番勝負「霊長化弾幕変化」
			\item 弐番勝負「肉食化弾幕変化」
		\end{enumerate}
	\item コマンドカード
		\begin{enumerate}
			\item 
			\item 
			\item 
			\item 
			\item 
			\item 
			\item 
			\item 
			\item 
			\item 
		\end{enumerate}
\end{itemize}
\pagebreak

\subsection{Extra スターター}
Ch: 20, SC: 20, Cm: 10
\begin{itemize}
	\item キャラクターカード
		\begin{enumerate}
			\item 伊吹萃香
			\item 永江衣玖
			\item 比那名居天子
			\item 大ナマズ
			\item 秦こころ
			\item 宇佐見菫子
			\item 依神女苑
			\item 依神紫苑
			\item 茨木華扇
			\item 宇佐見蓮子
			\item マエリベリー・ハーン
			\item 非想天則
			\item 黄帝
			\item 久米
			\item 務光
			\item 竿打
			\item 彭祖
			\item 万歳楽
			\item 龍神
			\item 蜃
		\end{enumerate}
	\item スペルカード
		\begin{enumerate}
			\item 鬼火「超高密度燐禍術」
			\item 「百万鬼夜行」
			\item 光珠「龍の光る眼」
			\item 雷符「エレキテルの龍宮」
			\item 要石「天空の霊石」
			\item 乾坤「荒々しくも母なる大地よ」
			\item ナマズ「ほらほら世界が震えるぞ?」
			\item ナマズ「液状化現象で大地も泥のようじゃ!」
			\item ナマズ「発電だって頑張っちゃうぞ?」
			\item 憂符「憂き世は憂しの小車」
			\item 憂面「杞人地を憂う」
			\item 虎符「両門の彭祖」
			\item 雷符「微速の務光」
			\item 銃符「3D プリンターガン」
			\item 念力「サイコキネシスアプリ」
			\item 念力「テレキネシス 電波塔」
			\item 不運「ようこそ極貧の世界へ」
			\item 財禍「プラックピジョン」
			\item 貧符「ミスチャンススキャッター」
			\item 「最凶最悪の極貧不幸神」
		\end{enumerate}
	\item コマンドカード
		\begin{enumerate}
			\item 
			\item 
			\item 
			\item 
			\item 
			\item 
			\item 
			\item 
			\item 
			\item 
		\end{enumerate}
\end{itemize}
\pagebreak

\subsection{輝針城+紺珠伝スターター}
Ch: 20, SC: 20, Cm: 10
\begin{itemize}
	\item キャラクターカード
		\begin{enumerate}
			\item わかさぎ姫
			\item 赤蛮奇
			\item 今泉影狼
			\item 九十九弁々
			\item 九十九八橋
			\item 鬼人正邪
			\item 少名針妙丸
			\item 堀川雷鼓
			\item 清蘭
			\item 鈴瑚
			\item ドレミー・スイート
			\item 稀神サグメ
			\item クラウンピース
			\item 純狐
			\item ヘカーティア・ラピスラズリ
			\item 稗田阿求
			\item 本居小鈴
			\item 管狐
			\item 運松
			\item 易者
		\end{enumerate}
	\item スペルカード
		\begin{enumerate}
			\item 水符「テイルフィンスラップ」
			\item 飛符「フライングヘッド」
			\item 牙符「月下の犬歯」
			\item 平曲「祇園精舎の鐘の音」
			\item 琴符「諸行無常の琴の音」
			\item 弦楽「嵐のアンサンブル」
			\item 欺符「逆針撃」
			\item 逆符「天下転覆」
			\item 小弾「小人の道」
			\item 妖剣「輝針剣」
			\item 一鼓「暴れ宮太鼓」
			\item 二鼓「怨霊アヤノツヅミ」
			\item 凶弾「スピードストライク」
			\item 兎符「ストロベリーダンゴ」
			\item 夢符「緋色の悪夢」
			\item 玉符「烏合の呪」
			\item 獄符「ヘルエクリプス」
			\item 「掌の純光」
			\item 異界「逢魔ガ刻」
			\item 月「ルナティックインパクト」
		\end{enumerate}
	\item コマンドカード
		\begin{enumerate}
			\item 
			\item 
			\item 
			\item 
			\item 
			\item 
			\item 
			\item 
			\item 
			\item 
		\end{enumerate}
\end{itemize}
\pagebreak

\subsection{天空璋+鬼形獣スターター}
Ch: 20, SC: 20, Cm: 10
\begin{itemize}
	\item キャラクターカード
		\begin{enumerate}
			\item 魂魄妖夢
			\item エタニティルラバ
			\item 坂田ネムノ
			\item 高麗野あうん
			\item 矢田寺成美
			\item 丁礼田舞
			\item 爾子田里乃
			\item 摩多羅隠岐奈
			\item 戎 瓔花
			\item 牛崎 潤美
			\item 庭渡 久侘歌
			\item 吉弔 八千慧
			\item 杖刀偶 磨弓
			\item 埴安神 袿姫
			\item 驪駒 早鬼
			\item オオカミ霊
			\item カワウソ霊
			\item オオワシ霊
			\item 饕餮
			\item 日焼けしたチルノ
		\end{enumerate}
	\item スペルカード
		\begin{enumerate}
			\item 空観剣「六根清浄斬」
			\item 蝶符「ミニットスケールス」
			\item 雨符「囚われの秋雨」
			\item 犬符「野良犬の散歩」
			\item 独楽「コマ犬回し」
			\item 魔符「インスタントボーディ」
			\item 竹符「バンブースピアダンス」
			\item 茗荷「フォゲットユアネーム」
			\item 後符「秘神の後光」
			\item 「アナーキーバレットヘル」
			\item 氷符「クールサンフラワー」
			\item 石符「ストーンウッズ」
			\item 水符「水配りの試練」
			\item 亀符「亀甲地獄」
			\item 埴輪「弓兵埴輪」
			\item 埴輪「騎馬兵埴輪」
			\item 方形「方形造形術」
			\item 方形「スクエアクリーチャー」
			\item 勁疾技「スリリングショット」
			\item 「フォロミーアンアフライド」
		\end{enumerate}
	\item コマンドカード
		\begin{enumerate}
			\item 
			\item 
			\item 
			\item 
			\item 
			\item 
			\item 
			\item 
			\item 
			\item 
		\end{enumerate}
\end{itemize}
\pagebreak

\subsection{封魔録+夢時空スターター}
Ch: 25, SC: 10, Cm: 15
\begin{itemize}
	\item キャラクターカード
		\begin{enumerate}
			\item レミリア・スカーレット
			\item フランドール・スカーレット
			\item 博麗霊夢
			\item 玄爺
			\item 神社戦車
			\item ふらわ〜戦車
			\item 里香
			\item 呪い子
			\item 明羅
			\item 幻夢盤
			\item 砲台
			\item 魔天使
			\item 飛行戦車イビルアイΣ
			\item エレン
			\item 小兎姫
			\item カナ・アナベラル
			\item 朝倉理香子
			\item 北白河ちゆり
			\item 岡崎夢美
			\item ま○ち
			\item る〜こと
			\item ミミちゃん
			\item 槌の子
			\item ソクラテス
			\item 先代巫女
		\end{enumerate}
	\item スペルカード
		\begin{enumerate}
			\item 無題「空を飛ぶ不思議な巫女」
			\item 夢境「二重大結界」
			\item 境界「二重弾幕結界」
			\item 宝符「躍る陰陽玉」
			\item 神術「吸血鬼幻想」
			\item 神槍「スピア・ザ・グングニル」
			\item 運命「ミゼラブルフェイト」
			\item 禁忌「クランベリートラップ」
			\item 禁忌「レーヴァテイン」
			\item 禁忌「恋の迷路」
		\end{enumerate}
	\item コマンドカード
		\begin{enumerate}
			\item 
			\item 
			\item 
			\item 
			\item 
			\item 
			\item 
			\item 
			\item 
			\item 
			\item 
			\item 
			\item 
			\item 
			\item 
		\end{enumerate}
\end{itemize}
\pagebreak

\subsection{靈異伝+怪綺談スターター}
Ch: 25, SC: 10, Cm: 15
\begin{itemize}
	\item キャラクターカード
		\begin{enumerate}
			\item 古明地さとり
			\item 古明地こいし
			\item アリス・マーガトロイド
			\item 神玉
			\item 幽幻魔眼
			\item 魅魔
			\item エリス
			\item キクリ
			\item サリエル
			\item 金伽羅
			\item 輪妖精
			\item サラ
			\item 神鏡
			\item ルイズ
			\item ルシア
			\item ユキ
			\item マイ
			\item 妖奈
			\item 夢子
			\item 神綺
			\item トランプキング
			\item 邪龍
			\item 沓面
			\item 蟒蛇
			\item 煙々羅
		\end{enumerate}
	\item スペルカード
		\begin{enumerate}
			\item 犠牲「スーサイドパクト」
			\item 人形「セミオートマトン」
			\item 魔符「アーティフルサクリファイス」
			\item 操符「マリオネットパラル」
			\item 想起「恐怖催眠術」
			\item 心花「カメラシャイローズ」
			\item 想起「うろおぼえの金閣寺」
			\item 記憶「DNA の瑕」
			\item 本能「イドの解放」
			\item 抑制「スーパーエゴ」
		\end{enumerate}
	\item コマンドカード
		\begin{enumerate}
			\item 
			\item 
			\item 
			\item 
			\item 
			\item 
			\item 
			\item 
			\item 
			\item 
			\item 
			\item 
			\item 
			\item 
			\item 
		\end{enumerate}
\end{itemize}
\pagebreak

\subsection{未割り当て}
エキスパンションなどで収録予定
\begin{itemize}
	\item ゴリアテ人形
	\item 座敷わらし
	\item ケセランパサラン
	\item 塩屋敷の旦那
	\item 京人形
	\item 露西亜人形
	\item 西藏人形
	\item オルレアン人形
	\item 仏蘭西人形
	\item 倫敦人形
	\item 和蘭人形
	\item 藁人形
	\item ティターニア
	\item 人間霊
	\item 埴輪
	\item サーヴァントフライヤー
	\item 宇治橋の怨霊
	\item 山童
	\item 天魔
	\item マミ
	\item 陰陽師
	\item いぬさくや
	\item れみにゃ
	\item ふにゃん
	\item メイベル
	\item アリオーシュ
	\item 小狐
	\item 囲碁少女
	\item お掃除メイド
	\item 神道国幻想
	\item 月光の反魂
	\item 未来世紀
	\item 蓮花の魔法使い
	\item Lotus Dream 霊夢
	\item 博麗霊夢の休日
	\item 霊夢のお茶会
	\item 最も好奇心の高い僕
	\item 最も美しいボク
	\item 最も幼い僕
	\item 最も臆病な僕
	\item 最も聡明な僕
	\item 最も大人びた僕
	\item 最も警戒心の強い僕
	\item 最も早起きな僕
	\item ビール子
	\item 仏少女
	\item トップ絵の少女
\end{itemize}

\onecolumn
\section{クレジット}
\begin{tabbing}
	\hspace{15\zw}\=\hspace{15\zw}\=\kill\\
	Total Design\>さーくる⑨\\
	Planning\>ひとみさん(さーくる⑨)\\
	Game Design\>ひとみさん(さーくる⑨)\\
	Adjusting Adviser\>SHU\\
	\>さんろ\\
	Design Adviser\>W・F(さーくる⑨)\\
	\>SHU\\
	Visual Assistant\>湯島\\
	\TeX nical Assistant\>Why Hitomisan use \TeX\ to develop this game?\\
	\hmemph{さーくる⑨ Member}\>ひとみさん(代表)\\
	\>W・F\\
	\>Ocean@海P\\
	\>kalna\\
	\>NEO\\
	\hmemph{Illustrator}\>\hmemph{募集中}\\
	湯島\\
	\hmemph{Fonts}\\
	{\fontspec[Scale=.85]{NKD04_Playing_Cards_Index}NKD04 Playing Card's Index}\>
		\url{http://hwm3.gyao.ne.jp/shiroi-niwatori/nishiki-teki.htm}\\
	{\jfontspec{nishiki-teki}にしき的フォント(絵文字)}\>
		〃\\
	{\fontspec{FOT-RodinNTLGPro-M}LETS Fonts}\>
		\url{https://lets-site.jp/}\\
	{\fontspec{kawoszeh}kawoszeh}\>
		\url{http://www.glukfonts.pl/font.php?font=Kawoszeh}\\
	Hall Fetica Familly\>
		\url{http://moorstation.org/typoasis/designers/gemnew/home.htm}\\
		
\end{tabbing}

\doclicenseThis
\end{document}

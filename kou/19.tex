\documentclass[
	jafontsize=1.8mm,
	baselineskip=1.1zh,
	cmyk,
	paper={63mm,88mm},
	gutter=0pt,
	head_space=0pt,
	foot_space=0pt,
	open_bracket_pos=zenkakunibu_nibu,
]{jlreq}
\usepackage{luatexja,xparse,fontspec,bxokumacro,bxghost,ifthen}
\ltjsetparameter{jacharrange={-3,-8}}
\usepackage[no-math,match,deluxe]{luatexja-preset}
\usepackage[x-1a]{pdfx}

\usepackage[pdfbox]{gentombow}
\settombowbleed{3mm}

\usepackage{tikz}
\usetikzlibrary{calc,fadings,patterns,arrows.meta,shapes.symbols}
\setlength{\parskip}{0pt}
\usepackage[export]{adjustbox}

\usepackage{luacode}
%% \PUACHAR{«num»}
% コードポイントが num である文字を出力する
\begin{luacode*}
	my = my or {}
	function my.pua(n)
		local t, c = font.getfont(font.current()), 0
		if t and t.shared and t.shared.rawdata and t.shared.rawdata.descriptions then
			for i,v in pairs(t.shared.rawdata.descriptions) do
				if not v.name and v.unicode==n then c=i; break end
			end
		end
		tex.sprint(tostring(c))
	end
\end{luacode*}
\def\PUACHAR#1{\char\directlua{my.pua(\the\numexpr#1)}}

\usepackage{pdfrender}
\NewDocumentCommand\hmfukuro{O{.2pt} +m O{white}}{\textpdfrender{
	TextRenderingMode=FillStroke,
	LineWidth=#1,
	FillColor=#3,
}{#2}}

\newlength{\OrigWidth}
%% \hminwidth[*]{«横幅»}{«テキスト»}
% テキストを指定の横幅に"イイカンジに"出力する
% *-> 均等割り付けしない
\NewDocumentCommand\hminwidth{s m m}{{%
	\settowidth{\OrigWidth}{#3}% \settowidth 便利
	\ifthenelse{\lengthtest{\OrigWidth > #2}}{%
		% 指定幅より大きい場合は横方向に縮小する
		\resizebox{#2}{\height}{#3}% 横幅だけ変える
	}{%else
		\IfBooleanTF{#1}{%
			% * 付きで指定幅以下ならそのまま出力
			#3
		}{%
			% 指定幅以下の場合は均等割りを行う
			\kintou{#2}{\jghostguarded{}#3\jghostguarded{}}%
		}%
	}%
}}

\newcommand{\©}{\jghostguarded{©}}
\newcommand{\・}{\mbox{・}}
\newcommand{\↓}{{\fontspec[Scale=0.86]{nishiki-teki}\rotatebox[origin=c]{90}{\PUACHAR{"293E}}}}
\catcode` \active \newcommand{ }{\mbox{\ }}
\newcommand{\hmzr}{\vphantom{あ}}

\NewDocumentCommand\α{s}{%
	\tikz[baseline=(T.base)]
	\IfBooleanTF{#1}{
		\node[inner sep=0pt,outer sep=0pt,signal,color=black,draw,black](T){自動α}
	}{
		\node[inner sep=0pt,outer sep=0pt,signal,color=white,fill=black](T){自動α}
	};%
}
\NewDocumentCommand\β{s}{%
	\tikz[baseline=(T.base)]
	\IfBooleanTF{#1}{
		\node[inner sep=0pt,outer sep=0pt,signal,color=black,draw,black](T){自動β}
	}{
		\node[inner sep=0pt,outer sep=0pt,signal,color=white,fill=black](T){自動β}
	};%
}
\NewDocumentCommand\γ{s}{%
	\tikz[baseline=(T.base)]
	\IfBooleanTF{#1}{
		\node[inner sep=0pt,outer sep=0pt,signal,color=black,draw,black](T){自動γ}
	}{
		\node[inner sep=0pt,outer sep=0pt,signal,color=white,fill=black](T){自動γ}
	};%
}
\NewDocumentCommand\自分{s m}{%
	\tikz[baseline=(T.base)]
	\IfBooleanTF{#1}{
		\draw node[inner sep=0pt,outer sep=0pt,left,color=black,draw,black](T){自手番\hmzr}
		node[inner sep=0pt,outer sep=0pt,right,signal,color=black,draw,black](T){\ #2\hmzr}
	}{
		\draw node[inner sep=0pt,outer sep=0pt,left,color=black,draw,black,fill=cyan](T){自手番\hmzr}
		node[inner sep=0pt,outer sep=0pt,right,signal,color=black,draw,black,fill=white](T){\ #2\hmzr}
	};%
}
\NewDocumentCommand\相手{s m}{%
	\tikz[baseline=(T.base)]
	\IfBooleanTF{#1}{
		\draw node[inner sep=0pt,outer sep=0pt,left,color=black,draw,black](T){相手番\hmzr}
		node[inner sep=0pt,outer sep=0pt,right,signal,color=black,draw,black](T){\ #2\hmzr}
	}{
		\draw node[inner sep=0pt,outer sep=0pt,left,color=black,draw,black,fill=pink](T){相手番\hmzr}
		node[inner sep=0pt,outer sep=0pt,right,signal,color=black,draw,black,fill=white](T){\ #2\hmzr}
	};%
}
\NewDocumentCommand\常時{s m}{%
	\tikz[baseline=(T.base)]
	\IfBooleanTF{#1}{
		\draw node[inner sep=0pt,outer sep=0pt,left,color=black,draw,black](T){常時\hmzr}
		node[inner sep=0pt,outer sep=0pt,right,signal,color=black,draw,black](T){\ #2\hmzr}
	}{
		\draw node[inner sep=0pt,outer sep=0pt,left,color=black,draw,black,fill=yellow](T){常時\hmzr}
		node[inner sep=0pt,outer sep=0pt,right,signal,color=black,draw,black,fill=white](T){\ #2\hmzr}
	};%
}
\NewDocumentCommand\収集{s}{%
	\tikz[baseline=(T.base)]
	\IfBooleanTF{#1}{
		\node[inner sep=0pt,outer sep=0pt,signal,color=black,draw,black](T)
			{収集}
	}{
		\node[inner sep=0pt,outer sep=0pt,signal,color=white,draw=black,fill=darkgray](T)
			{収集}
	};%
}

\pagestyle{empty}
\setlength{\parindent}{0\zw}

\setmainfont{FOT-MatissePro-M}\setmainjfont{FOT-MatissePro-M}
\setsansfont{FOT-RodinNTLGPro-DB}\setsansjfont{FOT-RodinNTLGPro-DB}
\newfontfamily\hmcfont[Scale=0.9]{NKD04_Playing_Cards_Index}
\newfontfamily\hmtfont[Scale=1.2]{kawoszeh}
\newfontfamily\hmgfont[Scale=0.9]{NKD04_Playing_Cards_Index}
\newfontfamily\hmgcfont[Scale=1.2]{kawoszeh}
\newjfontfamily\hmsjfont[Scale=1.0]{FOT-RodinNTLGPro-UB}
\newfontfamily\hmsafont[Scale=1.0]{FOT-RodinNTLGPro-UB}
\newcommand{\hmsfont}{\hmsafont\hmsjfont}
\newfontfamily\hmpfont[Scale=1.2]{Hall Fetica Upper}
\newfontfamily\hmnafont{FOT-TsukuOldGothicStd-B}
\newjfontfamily\hmnjfont{FOT-TsukuOldGothicStd-B}
\newcommand{\hmnfont}{\hmnafont\hmnjfont}
% \newfontfamily\hmxafont{FOT-TsukuOldGothicStd-B}
% \newjfontfamily\hmxjfont{FOT-TsukuOldGothicStd-B}
% \newcommand{\hmxfont}{\hmxafont\hmxjfont}
\newfontfamily\nishikiafont{nishiki-teki}
\newjfontfamily\nishikijfont{nishiki-teki}
\newcommand{\nishikifont}{\nishikiafont\nishikijfont}

\colorlet{nodecolor}{violet!90!blue}
\colorlet{costcolor}{olive!30!black}
\colorlet{spiritcolor}{red!70!black}
\colorlet{grazecolor}{yellow!70!black}
\colorlet{collectcolor}{gray!90}
\colorlet{rangebcolor}{black!90}
\colorlet{rangecolor}{black!10}
\colorlet{characolor}{teal!60}
\colorlet{spellcolor}{orange!60}
\colorlet{commandcolor}{gray!60}
\colorlet{incidentcolor}{magenta!70!black}
\colorlet{gridcolor}{cyan!60}
\definecolor{richblack}{cmyk}{.2,.2,.2,1}

\colorlet{Aicocolor}{violet}
\colorlet{Bicocolor}{orange}
\colorlet{Cicocolor}{magenta}
\colorlet{Dicocolor}{olive}
\colorlet{Eicocolor}{green}
\colorlet{Ficocolor}{lime}
\colorlet{Gicocolor}{red}
\colorlet{Hicocolor}{teal}
\colorlet{Iicocolor}{cyan}
\colorlet{Licocolor}{yellow}
\colorlet{Jicocolor}{gray}
\colorlet{Licocolor}{darkgray}

\newcommand{\Ahmico}{{\color{Aicocolor}\jfontspec{nishiki-teki}\fontspec{nishiki-teki}\PUACHAR{"0262F}}}
\newcommand{\Bhmico}{{\color{Bicocolor}\jfontspec{nishiki-teki}\fontspec{nishiki-teki}\PUACHAR{"1F3DA}}}
\newcommand{\Chmico}{{\color{Cicocolor}\jfontspec{nishiki-teki}\fontspec{nishiki-teki}\PUACHAR{"1F571}}}
\newcommand{\Dhmico}{{\color{Dicocolor}\jfontspec{nishiki-teki}\fontspec{nishiki-teki}\PUACHAR{"026F0}}}
\newcommand{\Ehmico}{{\color{Eicocolor}\jfontspec{nishiki-teki}\fontspec{nishiki-teki}\PUACHAR{"1F407}}}
\newcommand{\Fhmico}{{\color{Ficocolor}\jfontspec{nishiki-teki}\fontspec{nishiki-teki}\PUACHAR{"1F3A9}}}
\newcommand{\Ghmico}{{\color{Gicocolor}\jfontspec{nishiki-teki}\fontspec{nishiki-teki}\PUACHAR{"1F3F0}}}
\newcommand{\Hhmico}{{\color{Hicocolor}\jfontspec{nishiki-teki}\fontspec{nishiki-teki}\PUACHAR{"1F408}}}
\newcommand{\Ihmico}{{\color{Iicocolor}\jfontspec{nishiki-teki}\fontspec{nishiki-teki}\PUACHAR{"1F326}}}
\newcommand{\Jhmico}{{\color{Licocolor}\jfontspec{nishiki-teki}\fontspec{nishiki-teki}\PUACHAR{"1F318}}}
\newcommand{\Khmico}{{\color{Jicocolor}\jfontspec{nishiki-teki}\fontspec{nishiki-teki}\PUACHAR{"1F47B}}}
\newcommand{\Lhmico}{{\color{Licocolor}\jfontspec{nishiki-teki}\fontspec{nishiki-teki}\PUACHAR{"1F3E2}}}
\newcommand{\thmico}{{\jfontspec{nishiki-teki}\fontspec{nishiki-teki}\PUACHAR{"F015C}}}
\newcommand{\ohmico}{{\jfontspec{nishiki-teki}\fontspec{nishiki-teki}\PUACHAR{"1F300}}}
\newcommand{\ahmico}{{\jfontspec{nishiki-teki}\fontspec{nishiki-teki}\PUACHAR{"0F5C0}}}
\newcommand{\vhmico}{{\jfontspec{nishiki-teki}\fontspec{nishiki-teki}\PUACHAR{"0292D}}}
\newcommand{\ihmico}{{\jfontspec{nishiki-teki}\fontspec{nishiki-teki}\PUACHAR{"F00F1}}}
\newcommand{\mhmico}{{\jfontspec{nishiki-teki}\fontspec{nishiki-teki}\PUACHAR{"F01CF}}}
\newcommand{\chmico}{{\jfontspec{nishiki-teki}\fontspec{nishiki-teki}\PUACHAR{"1F3AD}}}
\newcommand{\ehmico}{{\jfontspec{nishiki-teki}\fontspec{nishiki-teki}\PUACHAR{"02694}}}
\newcommand{\whmico}{{\jfontspec{nishiki-teki}\fontspec{nishiki-teki}\PUACHAR{"1F342}}}
\newcommand{\shmico}{{\jfontspec{nishiki-teki}\fontspec{nishiki-teki}\PUACHAR{"1F54B}}}
\newcommand{\xhmico}{{\jfontspec{nishiki-teki}\fontspec{nishiki-teki}\PUACHAR{"1F7B0}}}

\newcommand{\hmmimg}{}
\newcommand{\hmmname}{レーベルのあの子}
\newcommand{\hmmnamer}{れーべるのあのこ}
\newcommand{\hmmtype}{Ch}
\newcommand{\hmmnode}{5}
\newcommand{\hmmcost}{3}
\newcommand{\hmmspirit}{0}
\newcommand{\hmmgraze}{3}
\newcommand{\hmmrace}{人間}
\newcommand{\hmmplace}{F}
\newcommand{\hmmtext}{%
	\常時{0}〔このキャラクター〕と〔あなたの場の「ジャケットのあの子」1 枚〕を
	決死状態にする。その後、〔キャラクター 3 枚まで〕を選んで決死状態にする。
}
\newcommand{\hmmp}{5}
\newcommand{\hmmt}{5}
\newcommand{\hmmflavor}{}
\newcommand{\hmmnumtype}{Dr.}
\newcommand{\hmmnum}{19}
\newcommand{\hmmexpantion}{<Sk>}
\newcommand{\hmmill}{}

\providecommand{\hmmimg}{}
\providecommand{\hmmfullimg}{}
\providecommand{\hmmname}{}
\providecommand{\hmmnamer}{}
\providecommand{\hmmtype}{}
\providecommand{\hmmnode}{}
\providecommand{\hmmcost}{}
\providecommand{\hmmspellist}{}
\providecommand{\hmmspirit}{}
\providecommand{\hmmgraze}{}
\providecommand{\hmmcollect}{}
\providecommand{\hmmrangea}{}
\providecommand{\hmmranget}{}
\providecommand{\hmmrace}{}
\providecommand{\hmmplace}{}
\providecommand{\hmmtext}{}
\providecommand{\hmmp}{-}
\providecommand{\hmmt}{-}
\providecommand{\hmmflavor}{}
\providecommand{\hmmcname}{}
\providecommand{\hmmcrace}{}
\providecommand{\hmmcspirit}{}
\providecommand{\hmmcgraze}{}
\providecommand{\hmmcp}{}
\providecommand{\hmmct}{}
\providecommand{\hmmctext}{}
\providecommand{\hmmnumtype}{}
\providecommand{\hmmnum}{}
\providecommand{\hmmexpantion}{}
\providecommand{\hmmill}{}

\ifthenelse{\equal{\hmmtype}{Ch}}{
	\colorlet{hmmtypecol}{characolor}
	\providecommand{\hmmback}{ch_back.pdf}
}{\ifthenelse{\equal{\hmmtype}{SC}}{
	\colorlet{hmmtypecol}{spellcolor}
	\providecommand{\hmmback}{sc_back.pdf}
}{\ifthenelse{\equal{\hmmtype}{Cm}}{
	\colorlet{hmmtypecol}{commandcolor}
	\providecommand{\hmmback}{cm_back.pdf}
}{\ifthenelse{\equal{\hmmtype}{In}}{
	\colorlet{hmmtypecol}{incidentcolor}
	\providecommand{\hmmback}{in_back.pdf}
}{}}}}

\edef\hmmmplace{\expandafter\noexpand\expandafter\csname\hmmplace hmico\endcsname}
\edef\hmmplacecol{\hmmplace icocolor}
\colorlet{hmmplacecol}{\hmmplacecol}

\edef\hmmmranget{\expandafter\noexpand\expandafter\csname\hmmrangea hmico\endcsname}
\edef\hmmmrangea{\expandafter\noexpand\expandafter\csname\hmmranget hmico\endcsname}

\newif\ifhmin
\newif\ifhmfill
\newif\ifhmgr
\newif\ifhmctxt
\ifthenelse{\NOT\equal{\hmmgraze}{}}{\hmgrtrue}{\hmgrfalse}
\ifthenelse{\NOT\equal{\hmmcname}{}}{\hmctxttrue}{\hmctxtfalse}
\ifthenelse{\NOT\equal{\hmmfullimg}{}}{\hmfilltrue\edef\hmmimg{\hmmfullimg}}{\hmfillfalse}
\ifthenelse{\NOT\equal{\hmmimg}{}}{}{\def\hmmimg{uni.pdf}}
\ifthenelse{\equal{\hmmtype}{In}}{\hmintrue}{\hminfalse}

\begin{document}
\noindent
\begin{tikzpicture}[x=1mm,y=1mm]
	\setlength{\parindent}{1\zw}
	% 座標
	\coordinate(OL1)at(63,88);\coordinate(OL3)at(0,0);
	\coordinate(OOL1)at($(OL1)+(3,3)-(0,.71pt)$);
	\coordinate(OOL3)at($(OL3)-(3,3)-(0,.71pt)$);
	\coordinate(N1)at(10,86);\coordinate(N3)at(2,78);
	\coordinate(N2)at(N1-|N3);\coordinate(N4)at(N1|-N3);
	\coordinate(C1)at(16,82);\coordinate(C3)at(8,74);
	\coordinate(C2)at(C1-|C3);\coordinate(C4)at(C1|-C3);
	\coordinate(TC)at(12,84);
	\coordinate(SC)at(49,82);\coordinate(GC)at(56,81);
	\coordinate(SC2)at(51,82);\coordinate(GC2)at(57,82);
	\coordinate(B1)at(59.5,30);\coordinate(B3)at(3.5,6);
	\coordinate(B3*)at($(B3)+(0,9)$);
	\coordinate(I1)at(59,84);\coordinate(I3)at(10,32);
	\coordinate(O1)at(61,86);\coordinate(O3)at(2,2);
	\coordinate(MT)at(5,78);\coordinate(M1)at(10,78);\coordinate(M3)at(2,30);
	\coordinate(RT)at(9,74);
	% 枠
	% 外枠
	\useasboundingbox(OL1)rectangle(OL3);
	\draw(O1)rectangle(O3);
	\fill[richblack](OOL1)-|(OOL3)-|cycle (O1)|-(O3)|-cycle;
	\draw[path picture={\node at (path picture bounding box.center)
		{\includegraphics[height=88mm,clip]{../../ill/\hmmback}};}](O1)-|(O3)-|cycle;
	\ifhmfill
	%% フルアート
	\draw[path picture={\node at (path picture bounding box.center)
		{\includegraphics[width=66mm,height=94mm,min size={66mm}{94mm},{../../ill/\hmmimg}};}]
		(OOL1)rectangle(OOL3);
	\else
	% イラスト
	\draw[path picture={\node at (path picture bounding box.center)
		{\includegraphics[width=49mm,height=52mm,keepaspectratio,min size={49mm}{52mm}]{../../ill/\hmmimg}};}]
		(I1)rectangle(I3);
	\fi
	% カード名
	\fill[black,path fading=west](M1)rectangle(M3);
	\path(MT)node[below,inner sep=0,outer sep=0,font=\fontsize{6mm}{0}\hmnfont]
		{\hbox{\tate\hminwidth{48mm}{\hmfukuro[.3pt]{\hmmname}}}};
	\path(RT)node[below,white,inner sep=0,outer sep=0,font=\fontsize{2mm}{0}\hmnfont]
		{\hbox{\tate\hminwidth*{44mm}{\hmmnamer}}};
	% タイプ、術者
	\filldraw[fill=hmmtypecol!50,draw=hmmtypecol](TC)circle[radius=1.9];
	\draw(TC)node{\fontsize{2mm}{0}\hmtfont{\hmmtype}};
	\path(TC)--++(2,0)node[right,font=\large\hmsfont]{\fontsize{2.3mm}{0}\hmfukuro{\hmmspellist}};
	% コスト
	\filldraw[fill=costcolor!30,draw=costcolor,thick](C1)rectangle(C3);
	\fill[fill=costcolor!70]
		($(C1)!.5!(C2)$)--($(C2)!.5!(C3)$)--($(C3)!.5!(C4)$)--($(C4)!.5!(C1)$)--cycle
		(C1)-|(C3)|-cycle;
	\path(C1)|-(C3)node[near end,above,outer ysep=0,font=\fontsize{1.7mm}{0}\hmgcfont]{COST};
	\path(C1)--(C3)node[midway,font=\fontsize{5.8mm}{0}\hmcfont]{\hmfukuro{\strut\hmmcost}};
	% ノード
	\filldraw[fill=nodecolor!30,draw=nodecolor,thick](N1)rectangle(N3);
	\fill[fill=nodecolor!70](N1)--(N3)--(N4)--(N2)--cycle;
	\path(N1)-|(N3)node[near start,below,outer ysep=0,font=\fontsize{1.7mm}{0}\hmgcfont]{NODE};
	\path(N1)--(N3)node[midway,font=\fontsize{5.8mm}{0}\hmcfont]{\hmfukuro{\strut\hmmnode}};
	\ifhmgr
	% 霊力
	\filldraw[fill=spiritcolor!40,draw=spiritcolor,thick](SC)circle[radius=3];
	\fill[fill=spiritcolor!10](SC)circle[radius=2];
	\path(SC)--++(0,2.5)node[font=\fontsize{1.7mm}{0}\hmgcfont]{\hmfukuro{\strut Spirit}};
	\path(SC)node[font=\fontsize{4mm}{0}\hmgfont]{\hmmspirit};
	% グレイズ
	\fill[fill=grazecolor!10](GC)circle[radius=4.5];
	\fill[fill=grazecolor!40](GC)--++(-45:4.5)arc(315:135:4.5)--cycle;
	\draw[draw=grazecolor,thick](GC)circle[radius=4.5];
	\path(GC)--++(0,3.5)node[font=\fontsize{1.7mm}{0}\hmgcfont]{\hmfukuro{\strut Graze}};
	\path(GC)node[font=\fontsize{5.7mm}{0}\hmgfont]{\hmmgraze};
	\else\ifhmin
	% 収集条件
	\fill[fill=collectcolor!70](GC)circle[radius=4.5];
	\fill[fill=collectcolor!30]
		($(GC)+(60:4.5)$)arc(60:120:4.5)--($(GC)+(-120:4.5)$)arc(-120:-60:4.5)--cycle;
	\draw[draw=collectcolor,thick](GC)circle[radius=4.5];
	\path(GC)--++(0,3.5)node[font=\fontsize{1.7mm}{0}\hmgcfont]{\hmfukuro{\strut Collect}};
	\path(GC)node[font=\fontsize{5.7mm}{0}\hmgfont]{\hmmcollect};
	\else
	% 発動期間
	\filldraw[fill=rangebcolor,draw=rangebcolor!80,thick](SC2)circle[radius=2.5];
	\path(SC2)node[rangecolor,font=\fontsize{3.9mm}{0}]{\strut\hmmmranget};
	% 効果範囲
	\filldraw[fill=rangebcolor,draw=rangebcolor!80,thick](GC2)circle[radius=2.5];
	\path(GC2)node[rangecolor,font=\fontsize{3.9mm}{0}]{\strut\hmmmrangea};
	\fi\fi
	% 種族
	\path(I3)-|(I1)node[midway,above left,font=\large\hmsfont]{\hmfukuro{\hmmrace}};
	% テキスト
	\draw[fill=hmmplacecol!30,fill opacity=.85](B1)rectangle(B3);
	%% 場所
	\path(B1)--(B3)node[midway,text opacity=.3,font=\fontsize{20mm}{0}]{\hmmmplace};
	\ifhmctxt
	%% 変身用
	%% テキスト
	\path(B1)-|(B3*)node[
		midway,below right,outer sep=0,inner sep=1mm,text width=54mm,text justified,font=\sffamily]{
			\hmmtext
		};
	%% 戦闘力
	\path($(B1)-(7,0)$)|-(B3*)node
		[midway,above,outer sep=0,inner xsep=0,inner ysep=1mm,text width=14mm,text centered,font=\fontsize{4mm}{0}\hmpfont]
		{\mbox{\makebox[4mm][r]{\hmmp}/\makebox[4mm][l]{\hmmt}}};
	%% フレーバーテキスト
	\path(B3*)node[above right,outer sep=0,inner sep=1mm,text width=39mm,text justified,font=\fontsize{1.3mm}{1.43mm}]{
			\hmmflavor
		};
	%% 変身状態
	\draw(B3*)--(B1|-B3*)[thick];
	\path(B3*)node[below right,inner sep=0,outer sep=0]{\rotatebox{180}{%
		\begin{tikzpicture}[x=1mm,y=1mm]
			\coordinate(Q1)at(55,9);\coordinate(Q3)at(0,0);
			\useasboundingbox(Q1)rectangle(Q3);
			%% 名称、種族
			\path(Q1-|Q3)node[below right,inner sep=1,outer sep=1]
				{\Large\hmnfont\hmmcname—\small\hmsfont\hmmcrace};
			%% 霊力、グレイズ
			\path(Q1)node[below left,inner sep=1]{
				\hmgcfont Sp:%
				\begin{tikzpicture}[baseline=(T.base),x=1mm,y=1mm,inner sep=0,outer sep=0,anchor=center]
					\coordinate(T)at(0,0);
					\filldraw[fill=spiritcolor!40,draw=spiritcolor,thick](T)circle[radius=1.5];
					\fill[fill=spiritcolor!10](T)circle[radius=1];
					\path(T)node[font=\large\hmgfont]{\hmmcspirit};
				\end{tikzpicture}
				Gr:%
				\begin{tikzpicture}[baseline=(T.base),x=1mm,y=1mm,inner sep=0,outer sep=0,anchor=center]
					\coordinate(T)at(0,0);
					\fill[fill=grazecolor!10](T)circle[radius=1.8];
					\fill[fill=grazecolor!40](T)--++(-45:1.8)arc(315:135:1.8)--cycle;
					\draw[draw=grazecolor,thick](T)circle[radius=1.8];
					\path(T)node[font=\Large\hmgfont]{\hmmcgraze};
				\end{tikzpicture}
			};
			%% 戦闘力
			\path($(Q1)-(7,0)$)|-(Q3)node
				[midway,above,outer sep=0,inner xsep=0,inner ysep=1,text width=14mm,text centered,font=\fontsize{4mm}{0}\hmpfont]
				{\mbox{\makebox[4mm][r]{\hmmcp}/\makebox[4mm][l]{\hmmct}}};
			%% テキスト
			\path(Q3)node[above right,outer sep=1,inner sep=1,text width=39.6mm,text justified,font=\sffamily]{
				\hmmctext
			};
		\end{tikzpicture}%
	}};
	\else
	%% テキスト
	\path(B1)-|(B3)node
		[midway,below right,outer sep=0,inner sep=1mm,text width=54mm,text justified,font=\sffamily]{
			\hmmtext
		};
	%% 戦闘力
	\path($(B1)-(7,0)$)|-(B3)node
		[midway,above,outer sep=0,inner xsep=0,inner ysep=1mm,text width=14mm,text centered,font=\fontsize{4mm}{0}\hmpfont]
		{\mbox{\makebox[4mm][r]{\hmmp}/\makebox[4mm][l]{\hmmt}}};
	%% フレーバーテキスト
	\path(B3)node[above right,outer sep=0,inner sep=1mm,text width=39mm,text justified,font=\fontsize{1.3mm}{1.43mm}\selectfont,]{
			\hmmflavor
		};
	\fi
	% フッター
	\path(O3)node[above right,text width=25\zw,align=left,inner ysep=0,outer ysep=0]
		{\hmsfont\fontsize{1.2mm}{1}\selectfont%
			\hmfukuro{{\©}上海アリス幻樂団\\
			{\©}2020-2020 さーくる⑨ CC-BY-NC 4.0}};
	\path(O1)|-(O3)node[pos=.75,above left,inner xsep=0,outer xsep=0,font=\hmsfont\fontsize{1.8mm}{0}\hmsfont]
		{\hmfukuro{\hmmnumtype~}};
	\path(O1)|-(O3)node[pos=.75,above right,inner xsep=0,outer xsep=0,font=\hmsfont\fontsize{1.8mm}{0}\hmsfont]
		{\hmfukuro{\hmmnum}};
	\path(O1)|-(O3)node[pos=.68,above,font=\hmsfont\fontsize{1.8mm}{0}\hmsfont]{\hmfukuro{\hmmexpantion}};
	\path(O1)|-(O3)node[midway,above left,font=\hmsfont\fontsize{1.8mm}{0}\hmsfont]{\hmfukuro{Ill. \hmmill}};
	% % 方眼
	% \draw[very thin,gridcolor,nearly transparent](OL1)grid[step=1](OL3);
	% \draw[gridcolor,nearly transparent]($(OL1)-(2,2)$)rectangle($(OL3)+(2,2)$);
	% \draw[thin,gridcolor,nearly transparent](OL1)grid[step=10](OL3);
	% \draw[thin,gridcolor,nearly transparent](OL1)rectangle(OL3);
\end{tikzpicture}%
\end{document}
